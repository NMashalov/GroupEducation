Такие функции позволяю оценить по индивидуальным качествам учащихся их совместный вклад в дело. Примерами супераддитивных функций являются \begin{itemize}
    \item min-sum $\sum_{i} min([\vec{x}]_i,s^*)$, гдe $s^*$ - порог отсечки
    \item max-mean $N \cdot \bar{x} + \max_i(\vec{x} - \bar{x})$
    \item квадратичная форма $\vec{x}^T A \vec{x}$
\end{itemize}


Для квадратичной формы $\varepsilon^T \Lambda \varepsilon$ с
вектором распределенным нормально $\mathbf{\varepsilon} \sim \mathcal{N}(\mathbf{\mu},\mathbf{\Sigma})$ выполняются

\begin{equation}
    \begin{aligned}
        & \mathbb{E}[\varepsilon^T \Lambda \varepsilon] = \operatorname{tr}[\Lambda \Sigma] + \mu^T \Lambda \mu \\ 
        & \operatorname{\varepsilon^T \Lambda \varepsilon} = 2  \operatorname{tr}[\Lambda \Sigma] 
    \end{aligned}
\end{equation}

Доказательство утвереждения приведено в \cite{mathai1992quadratic}.