
Использование приближение среднего поля к задачам моделирования
распределения \cite{toscani2016kinetic}.


Применение физических методов в социальных и экономических задачах 
распостраненная практика исследований: \begin{itemize}
    \item экономическое моделирование \cite{carmona2020applications}
\end{itemize}

Приближение среднего актуально для постановок, в котором 
агенты слабо взаимодействуюит


Простейшей постановкой является проблема поселяющего (\textit{англ.} dean problem).
Группе студентов с попарными предпочтениями $g_{ij}$ необходимо распределить в группы, достигнув
максимума функции социального блага:
\begin{equation}
    C(\sigma) = \sum_{i<j} \g_{ij} \sigma_i \sigma_j
\end{equation}
В простейшем случае предполагается, что элементы вектора 
$\mathbf{\sigma}=(\simga_1,\dots,\sigma_n)$ распределения задаются 
в виде бинарной величины, что позволяет переписать уравнение как:
\begin{equation}
    C(\sigma) = \sum_{i \sim j } g_{ij} - \sum_{i \cancel{\sim} j } g_{ij}  
\end{equation}


В простейшем случае предполагается, что предпочтения студентов
распределены нормально $g_{ij} \sim \mathcal{N}(0,\sigma^2)$.

Для заданной постановки получены асимптотические постановки 
при $N \righarrow \infty$: 
\begin{equation}
    с N^{3/2} \le \mathrm{E} M_N \le C N^{3/2}
\end{equation}
Григорио Париси, первооткрывателем нарушения симметрии реплик, в 1979 
была получена точная оценка \cite{parisi1979infinite}:
\begin{equation}
    \lim_{N \rightarrow \infty} \frac{1}{N^{3/2}} \max_\sigma \sum_{i<j} g_{ij} \sigma_i \sigma_j = 0.76-[
\end{equation}


Вывод уравнения Гибрата \cite{toscani2016kinetic}.
Закон Гибрата известный экономический постулат, заключающий
что скорость роста фирмы пропорционально ее размеру \cite{france2001inegalites}:
$$
    x(\tau +1) = x(\tau) + \eta(\tau) x(\tau),
$$

\begin{equation}
    \frac{d}{d \tau} \int_{\mathrm{R_+}} \phi(x) f(x,\tau )= \lambda \mathrm{E}_\tau \int_{R_+} (\phi(x^*) - \phi(x)) f(x,\tau) dx \>
\end{equation}
где $\phi$ - обобщенная функция, $\lambda$ - задает частоту взаимодействия

Перейдем к фурье-образу функции плотности:
\begin{equation}
    \hat{f}(\xi,\tau) = \int_{\mathrm{R}^_+} f(x,\tau) e^{-i \xi x} dx
\end{equation}

Итоговое уравнение запишется как:
\begin{equation}
    \frac{\partial}{\partial \tau} \hat{f}(\xi,\tau) = \lambda (\mathrm{E}_\tau f(1+\eta,
    tau) - f(\xi,\tau))
\end{equation}

Для обобщенных функций вида $\phi(x)=x^n$ получаем, что 
моменты распределения запишутся как:
\begin{equation}\
    \frac{\partial}{\partial \tau} m_n(\tau) = \lambda \mathrm{E}_\tau (1+eta(\tau))^n -1) m_n(\tau)
\end{equation}