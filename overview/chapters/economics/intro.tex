Образование несет важную роль в экономике стран пост-индустриального развития.

В разделе описаны подходы к модельному исследованию командной деятельности.

Аналитическое изучение кооперативной деятельности выполняется в теории совместных игр 
В этом случае результирующий вклад задается агрегирующий усилия игроков функцией.
$$
 f_i(s) = \tilde{f_i} \left(s_i, \sum_{j=1}^n s_j \right),
$$

Классическим примером производственной функции является функция Кобба Дугласа

$$
    Q = A L^\beta K^\alpha,
$$
где \begin{itemize}
    \item $Q$ - общий объем производства;
    \item $L$ - вложенный труд; 
    \item $K$ - денежный капитал;
    \item $A$ - мультипликатор производства;
    \item $\alpha$ и $\beta$ - параметры эластичности выпуска.
\end{itemize}


Постоянная эластичность замещения обобщает функцию 
Кобба-Дугласа \cite{mcfadden1963constant}:
$$
    Q = A (a \cdot K^p + (1-a)\cdot L^p)^{\frac{v}{p}},
$$
где \begin{itemize}
    \item $\rho$ - параметр замещения; 
    \item $\alpha$ и $\beta$ - параметры эластичности выпуска.
\end{itemize}

При $\rho \rightarrow 0$ функция вырождается в функцию Кобба-Дугласа.

Случай многих аргументов запишется как:
$$
    Q \sum_i \left(\alpha_i x_i^\rho \right)^{\frac{\beta}{\rho}}, \sum_i \alpha_i=1
$$




