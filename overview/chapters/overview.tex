Совместным обучением (\textit{англ.} collaborative learning) называется подход, 
предполагающий выполнение образовательного задания командой обучающихся. Такой подход отличается активным взаимодействием обучающихся,
заключающийся в совместном поиске и анализе информации, разделение обязанностей по исполнению задач
и определение общей цели. Внедрение подхода совместного обучения в общее образование было выполнено 
Джеймса Коулмана \cite{coleman1974youth}. При составлении методических пособий использовался теория
социально-психологические основы организации группового обучения, описанная в 
работах педагога и психолога Теодора Ньюкомба \cite{newcomb1953approach}. К текущему моменту 
практика широко распространена в обучение музыке, инностранному языку и
 риторике \cite{mazen2000transforming}.

Формами совместного обучения:\begin{enumerate}
    \item 

\end{enumerate}


Тема имеет богатое психолого-социологическое описание. Основателем направления считается 
русский психолог Лев Семенович Психолог, разработавший 
концепцию зоны ближайшего развития в теории высших психологических процессов \cite{vigotski2014}.
В предметной литературе разбираются и разрешаются проблемы вовлеченности учащихся \cite{rau1990humanizing},
отношений между учащимися \cite{flowers2015friendship} и 
оценки предметных проектов \cite{newcomb1953approach}.

Аналитическое модели области строятся на теория оценки индивидуальных тестировочных заданий. \cite{dillenbourg1999collaborative} 
Существующеи теории классического тестирования \cite{traub1997classical} и отклика (\textit{англ.} Item Reponse Theory(IRT))
\cite{lord2012applications}
определены модельные

C развитием информационных технологий становится популярно цифровое совместное обучение
(\textit{англ.} Computer-supported collaborative learning(CSCL)) \cite{strijbos2010assessment}. 
Основы создания приложений совместного деятельности были предложены (\textit{англ.} groupware) 
Дугласом Енгельбратом в работах \cite{engelbart1984collaboration} \cite{engelbart1992toward}.
В таких форматах учитель c использованием цифровых площадок
устанавливает формата общения, оценки задания и презентации проекта \cite{lu2010scaffolding}.
Ключевым преимуществом направление является его доступность
и возможность к общению со сверстниками и единомышленниками из любой точки мира
\cite{stahl2009practice}. 


В обзоре представлены основные подходы к описанию коллективной деятельности:\begin{itemize}
    \item кооперативные игры, теоретический подход к описанию \cite{roth1988shapley}
    \item агрегирующие игры и приближение среднего поля \cite{}
    \item  
\end{itemize}

Разобраны подходы к решению практических задач: \begin{itemize}
    \item распределение учащихся в группы, составление заданий для отбора;
    \item устранению нечестного поведения игроков путем 
    введения принципала и задания механизма оценки;
\end{itemize}
