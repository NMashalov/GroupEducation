\documentclass{article}
\usepackage[backend=biber,autolang=other, citestyle=numeric-comp, sorting=none]{biblatex}
\usepackage{graphicx}

\usepackage[utf8]{inputenc}
\usepackage{amsmath}
\usepackage{amsopn}
\usepackage[makeroom]{cancel}
\usepackage[T2A]{fontenc}
\usepackage[english, russian]{babel}


\addbibresource{ref.bib}


\begin{document}


\tableofcontents

\title{Количественные подходы к описанию совместных учебных проектов}

\maketitle

\date{\vspace{-12ex}}

Совместное обучение включено в практики профессионального развития как 
форма образования, совмещающая приобретение глубоких предметных знаний и умений 
работы в команде.
В обзоре приведены подходы к развитию изучения совместного обучения,
основанные на теории обучения. Предложен алгоритм адаптивного подбора сложности,
учитывающий взаимодействие в группах.


\section{Введение}

Совместным обучением (\textit{англ.} collaborative learning) называется подход, 
предполагающий выполнение образовательного задания командой обучающихся. Такой подход отличается активным взаимодействием обучающихся,
заключающийся в совместном поиске и анализе информации, разделение обязанностей по исполнению задач
и определение общей цели. Внедрение подхода совместного обучения в общее образование было выполнено 
Джеймса Коулмана \cite{coleman1974youth}. При составлении методических пособий использовался теория
социально-психологические основы организации группового обучения, описанная в 
работах педагога и психолога Теодора Ньюкомба \cite{newcomb1953approach}. К текущему моменту 
практика широко распространена в обучение музыке, инностранному языку и
 риторике \cite{mazen2000transforming}.

Формами совместного обучения:\begin{enumerate}
    \item 

\end{enumerate}


Тема имеет богатое психолого-социологическое описание. Основателем направления считается 
русский психолог Лев Семенович Психолог, разработавший 
концепцию зоны ближайшего развития в теории высших психологических процессов \cite{vigotski2014}.
В предметной литературе разбираются и разрешаются проблемы вовлеченности учащихся \cite{rau1990humanizing},
отношений между учащимися \cite{flowers2015friendship} и 
оценки предметных проектов \cite{newcomb1953approach}.

Аналитическое модели области строятся на теория оценки индивидуальных тестировочных заданий. \cite{dillenbourg1999collaborative} 
Существующеи теории классического тестирования \cite{traub1997classical} и отклика (\textit{англ.} Item Reponse Theory(IRT))
\cite{lord2012applications}
определены модельные

C развитием информационных технологий становится популярно цифровое совместное обучение
(\textit{англ.} Computer-supported collaborative learning(CSCL)) \cite{strijbos2010assessment}. 
Основы создания приложений совместного деятельности были предложены (\textit{англ.} groupware) 
Дугласом Енгельбратом в работах \cite{engelbart1984collaboration} \cite{engelbart1992toward}.
В таких форматах учитель c использованием цифровых площадок
устанавливает формата общения, оценки задания и презентации проекта \cite{lu2010scaffolding}.
Ключевым преимуществом направление является его доступность
и возможность к общению со сверстниками и единомышленниками из любой точки мира
\cite{stahl2009practice}. 


В обзоре представлены основные подходы к описанию коллективной деятельности:\begin{itemize}
    \item кооперативные игры, теоретический подход к описанию \cite{roth1988shapley}
    \item агрегирующие игры и приближение среднего поля \cite{}
    \item  
\end{itemize}

Разобраны подходы к решению практических задач: \begin{itemize}
    \item распределение учащихся в группы, составление заданий для отбора;
    \item устранению нечестного поведения игроков путем 
    введения принципала и задания механизма оценки;
\end{itemize}


\section{Теория игр}

Образование несет важную роль в экономике стран пост-индустриального развития.

В разделе описаны подходы к модельному исследованию командной деятельности.

Аналитическое изучение кооперативной деятельности выполняется в теории совместных игр 
В этом случае результирующий вклад задается агрегирующий усилия игроков функцией.
$$
 f_i(s) = \tilde{f_i} \left(s_i, \sum_{j=1}^n s_j \right),
$$

Классическим примером производственной функции является функция Кобба Дугласа

$$
    Q = A L^\beta K^\alpha,
$$
где \begin{itemize}
    \item $Q$ - общий объем производства;
    \item $L$ - вложенный труд; 
    \item $K$ - денежный капитал;
    \item $A$ - мультипликатор производства;
    \item $\alpha$ и $\beta$ - параметры эластичности выпуска.
\end{itemize}


Постоянная эластичность замещения обобщает функцию 
Кобба-Дугласа \cite{mcfadden1963constant}:
$$
    Q = A (a \cdot K^p + (1-a)\cdot L^p)^{\frac{v}{p}},
$$
где \begin{itemize}
    \item $\rho$ - параметр замещения; 
    \item $\alpha$ и $\beta$ - параметры эластичности выпуска.
\end{itemize}

При $\rho \rightarrow 0$ функция вырождается в функцию Кобба-Дугласа.

Случай многих аргументов запишется как:
$$
    Q \sum_i \left(\alpha_i x_i^\rho \right)^{\frac{\beta}{\rho}}, \sum_i \alpha_i=1
$$







\subsection{Постановка}

Такие функции позволяю оценить по индивидуальным качествам учащихся их совместный вклад в дело. Примерами супераддитивных функций являются \begin{itemize}
    \item min-sum $\sum_{i} min([\vec{x}]_i,s^*)$, гдe $s^*$ - порог отсечки
    \item max-mean $N \cdot \bar{x} + \max_i(\vec{x} - \bar{x})$
    \item квадратичная форма $\vec{x}^T A \vec{x}$
\end{itemize}


Для квадратичной формы $\varepsilon^T \Lambda \varepsilon$ с
вектором распределенным нормально $\mathbf{\varepsilon} \sim \mathcal{N}(\mathbf{\mu},\mathbf{\Sigma})$ выполняются

\begin{equation}
    \begin{aligned}
        & \mathbb{E}[\varepsilon^T \Lambda \varepsilon] = \operatorname{tr}[\Lambda \Sigma] + \mu^T \Lambda \mu \\ 
        & \operatorname{\varepsilon^T \Lambda \varepsilon} = 2  \operatorname{tr}[\Lambda \Sigma] 
    \end{aligned}
\end{equation}

Доказательство утвереждения приведено в \cite{mathai1992quadratic}.



\subsection{Модель}

% \begin{figure}[h]
%     \centering
%     \includegraphics[width=0.5\textwidth]{assets/final/geometry.excalidraw.png}
%     \caption{Групповые задания}
%     \label{group_task}
% \end{figure}



\section{Экономический подход к описанию команд}


Образование несет важную роль в экономике стран пост-индустриального развития.

В разделе описаны подходы к модельному исследованию командной деятельности.

Аналитическое изучение кооперативной деятельности выполняется в теории совместных игр 
В этом случае результирующий вклад задается агрегирующий усилия игроков функцией.
$$
 f_i(s) = \tilde{f_i} \left(s_i, \sum_{j=1}^n s_j \right),
$$

Классическим примером производственной функции является функция Кобба Дугласа

$$
    Q = A L^\beta K^\alpha,
$$
где \begin{itemize}
    \item $Q$ - общий объем производства;
    \item $L$ - вложенный труд; 
    \item $K$ - денежный капитал;
    \item $A$ - мультипликатор производства;
    \item $\alpha$ и $\beta$ - параметры эластичности выпуска.
\end{itemize}


Постоянная эластичность замещения обобщает функцию 
Кобба-Дугласа \cite{mcfadden1963constant}:
$$
    Q = A (a \cdot K^p + (1-a)\cdot L^p)^{\frac{v}{p}},
$$
где \begin{itemize}
    \item $\rho$ - параметр замещения; 
    \item $\alpha$ и $\beta$ - параметры эластичности выпуска.
\end{itemize}

При $\rho \rightarrow 0$ функция вырождается в функцию Кобба-Дугласа.

Случай многих аргументов запишется как:
$$
    Q \sum_i \left(\alpha_i x_i^\rho \right)^{\frac{\beta}{\rho}}, \sum_i \alpha_i=1
$$







\subsection{Постановка}

Такие функции позволяю оценить по индивидуальным качествам учащихся их совместный вклад в дело. Примерами супераддитивных функций являются \begin{itemize}
    \item min-sum $\sum_{i} min([\vec{x}]_i,s^*)$, гдe $s^*$ - порог отсечки
    \item max-mean $N \cdot \bar{x} + \max_i(\vec{x} - \bar{x})$
    \item квадратичная форма $\vec{x}^T A \vec{x}$
\end{itemize}


Для квадратичной формы $\varepsilon^T \Lambda \varepsilon$ с
вектором распределенным нормально $\mathbf{\varepsilon} \sim \mathcal{N}(\mathbf{\mu},\mathbf{\Sigma})$ выполняются

\begin{equation}
    \begin{aligned}
        & \mathbb{E}[\varepsilon^T \Lambda \varepsilon] = \operatorname{tr}[\Lambda \Sigma] + \mu^T \Lambda \mu \\ 
        & \operatorname{\varepsilon^T \Lambda \varepsilon} = 2  \operatorname{tr}[\Lambda \Sigma] 
    \end{aligned}
\end{equation}

Доказательство утвереждения приведено в \cite{mathai1992quadratic}.



\subsection{Модель}

% \begin{figure}[h]
%     \centering
%     \includegraphics[width=0.5\textwidth]{assets/final/geometry.excalidraw.png}
%     \caption{Групповые задания}
%     \label{group_task}
% \end{figure}




\section{Развитие постановки}


Образование несет важную роль в экономике стран пост-индустриального развития.

В разделе описаны подходы к модельному исследованию командной деятельности.

Аналитическое изучение кооперативной деятельности выполняется в теории совместных игр 
В этом случае результирующий вклад задается агрегирующий усилия игроков функцией.
$$
 f_i(s) = \tilde{f_i} \left(s_i, \sum_{j=1}^n s_j \right),
$$

Классическим примером производственной функции является функция Кобба Дугласа

$$
    Q = A L^\beta K^\alpha,
$$
где \begin{itemize}
    \item $Q$ - общий объем производства;
    \item $L$ - вложенный труд; 
    \item $K$ - денежный капитал;
    \item $A$ - мультипликатор производства;
    \item $\alpha$ и $\beta$ - параметры эластичности выпуска.
\end{itemize}


Постоянная эластичность замещения обобщает функцию 
Кобба-Дугласа \cite{mcfadden1963constant}:
$$
    Q = A (a \cdot K^p + (1-a)\cdot L^p)^{\frac{v}{p}},
$$
где \begin{itemize}
    \item $\rho$ - параметр замещения; 
    \item $\alpha$ и $\beta$ - параметры эластичности выпуска.
\end{itemize}

При $\rho \rightarrow 0$ функция вырождается в функцию Кобба-Дугласа.

Случай многих аргументов запишется как:
$$
    Q \sum_i \left(\alpha_i x_i^\rho \right)^{\frac{\beta}{\rho}}, \sum_i \alpha_i=1
$$







\subsection{Постановка}

Такие функции позволяю оценить по индивидуальным качествам учащихся их совместный вклад в дело. Примерами супераддитивных функций являются \begin{itemize}
    \item min-sum $\sum_{i} min([\vec{x}]_i,s^*)$, гдe $s^*$ - порог отсечки
    \item max-mean $N \cdot \bar{x} + \max_i(\vec{x} - \bar{x})$
    \item квадратичная форма $\vec{x}^T A \vec{x}$
\end{itemize}


Для квадратичной формы $\varepsilon^T \Lambda \varepsilon$ с
вектором распределенным нормально $\mathbf{\varepsilon} \sim \mathcal{N}(\mathbf{\mu},\mathbf{\Sigma})$ выполняются

\begin{equation}
    \begin{aligned}
        & \mathbb{E}[\varepsilon^T \Lambda \varepsilon] = \operatorname{tr}[\Lambda \Sigma] + \mu^T \Lambda \mu \\ 
        & \operatorname{\varepsilon^T \Lambda \varepsilon} = 2  \operatorname{tr}[\Lambda \Sigma] 
    \end{aligned}
\end{equation}

Доказательство утвереждения приведено в \cite{mathai1992quadratic}.



\subsection{Модель}

% \begin{figure}[h]
%     \centering
%     \includegraphics[width=0.5\textwidth]{assets/final/geometry.excalidraw.png}
%     \caption{Групповые задания}
%     \label{group_task}
% \end{figure}




\printbibliography%


\end{document}




