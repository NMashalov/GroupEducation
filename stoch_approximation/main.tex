\documentclass{article}
\usepackage[T2A]{fontenc}
\usepackage[english, russian]{babel}

\begin{document}

\title{Изучение предметно следственных связей в естественном языке посредством графовых вероятностях моделей для адаптации к большим языковых моделей}

\section{Предисловие}

Стохастическая апппроксимация изучает 
проблемы оптимизации в условиях случайного отлика
заданного парой $(x,\xi)$, где $\x$ случайная величина
с плотностью вероятности $p_\xi(z)$
,как правило, предполагается несмещенность 
полученной $\mathbb{E}\xi=0$.

Базовой численной схемой решения является алгоритм
Роббинса-Монро \cite{monro1951}

В случае
известной функции ошибки оптимальный спуск 
в классе численных схем:

$$
    x^{k+1} = x^{k} - \gamma_k \phi(\nabla_x f(x^k,\xi^k)),
$$

Такие подходы называют \textit{псевдоградиентными} поскольку
выражение на градиент может быть задано неявно.

В работе \cite{поляк1980оптимальные} показан
оптимальный выбор функии $\phi(z)$ и шаг:
$$
   \phi(z) = \nabla^2 f(x_*)^{-1} J^{-1} \nabla \ln p_\xi (z)  
$$
Матрица Фишера записывается как :

$$
    J = \int \nabla \ln p_\xi(z) [\ln p_\xi(z)]^T p_\xi (z) dz
$$

ДЛя этого случая можно показать убывание дисперсии с шагом согласно ЦПТ:
$$
    \sqrt{N} (x^N - x_*) \in \mathcal{N}\left(0,[\nalba^2 f(x_*)]^{-1}J[\nalba^2 f(x_*)]^{-1}\right)
$$

При  $\phi(z)$ в смысле полуопределенного отношения частичного порядка

Схема Руперта-Поляка-Юдицкого \cite{поляк1990новый} 
\cite{ruppert1988efficient}.

Борис Теодорович Поляк предложил $\phi(z)=z$ 
и шаг как $\eta \in (\frac{1}{2},1)$

\cite{polyak1992acceleration} 

$$
    
$$




$$
    \mathbb{E} \| x \|
$$
