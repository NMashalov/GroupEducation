\documentclass[a4paper,11pt]{article}

\usepackage{cmap}
\usepackage[utf8]{inputenc}
\usepackage[T2A]{fontenc}
\usepackage[english, russian]{babel}
\usepackage{ragged2e}

\begin{document}

\title{Мотивационное письмо}
\date{\vspace{-12ex}}
\maketitle
\thispagestyle{empty}
Уважаемая комиссия!

Я бы хотел посетить школу для установления научных коммуникаций и углубления знаний в теории оптимизации
для развития научной работы. Современные подходы к моделированию постановок в сфере образования, такие 
как дизайн управленческих механизмов, байесовые рейтинг-системы и дифференциальные игры, используют для анализа 
намерений игроков продвинутые методы стохастической теории оптимизаций. В связи с этим мне особенно интересны научные работы и доклады,
использующие теорию мер и функциональный анализ.

В этом году я закончил диссертационную работу по методам применения больших языковых моделей в образовании. Результатом научной работы
стал персональный ассистент, обучающий игре шахматам игроков с разным уровнем навыка. Для адаптации сложности игры согласно ее результатам
был разработан стохастический алгоритм, совмещающий рейтинговую модель Эло и численную схему спуска по типу алгоритма Роббинса-Монро. Объединение идей было достигнуто 
путем модификации численной схемы по методу Дана Анбара для случая бернуллевского отклика, заданного логистической функцией. Численный эксперимент
установил, что предложенный алгоритм имеют большую скорость сходимости, что составляет научную ценность работы.

Весной этого года для апробации диссертационной работы я доложил две работы 
"Оценка влияния кредитных условий на конкурентные предложения малых поставщиков в сфере образования"
и "Разработка пакетного модуля ShuemacherOCR на языке Python
для работы с методической литературой" на 66-ой Всероссийской научной конференции МФТИ. Работы были приняты комиссией и на текущий
момент находятся в публикации. Также для расширения знаний в сфере машинного обучения я посетил смену по доверенному искусственному интеллекту в Сириусе,
где совместно с группой Александра Владимировича Гасникова работал над задачами федеративного обучения.
Базовые знания по работе и созданию порождающих моделей я приобрел на кафедре интеллектуального анализа данных 
под руководством Романа Исаченко, где с оценкой отлично выполнил все 
современные архитектуры для работы с изображениями как VAE, GAN,
нормализующие потоки и диффузионные модели.

Спасибо Вам за рассмотрение моей заявки!
\begin{FlushRight}
    Аспирант МФТИ

    Кафедра инновационной педагогики
    
    Машалов Никита
\end{FlushRight}

\end{document}